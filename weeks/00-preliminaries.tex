Discrete Random Variable

The capital X is commonly used to denote a random variable in probability theory. It can represent either a discrete or continuous random variable, depending on the context.
Let me explain how a discrete random variable works:
A discrete random variable X is a variable that can only take on a countable number of distinct values. These values are often (but not always) integers. Each possible value of X has an associated probability of occurring.
Key characteristics of discrete random variables:

Countable outcomes: The variable can only take on specific, separated values.
Probability mass function (PMF): This function, often denoted as P(X = x), gives the probability that X takes on a specific value x.
Cumulative distribution function (CDF): This function, denoted as F(x), gives the probability that X is less than or equal to x.
Discrete jumps: The CDF of a discrete random variable has "jumps" at the possible values of X.
Sum of probabilities: The sum of probabilities for all possible outcomes must equal 1.

Example:
Let's consider a simple example - rolling a fair six-sided die. Here, X could represent the outcome of the roll:

X can take values 1, 2, 3, 4, 5, or 6
The probability of each outcome is 1/6
P(X = 3) = 1/6 (probability of rolling a 3)
F(3) = P(X ≤ 3) = 1/2 (probability of rolling 3 or less)

Expectation
