\newcommand{\pmf}{p_X}

\begin{definition}
    Discrete Random Variable: 
\end{definition}

\begin{definition}
    Probability Mass Function of
    \[
    X \; \text{as} \;\;p_X:\Omega_X \to [0,1] \qquad  p_X(k) = P(X = k)
    \]
\end{definition}
\begin{itemize}
    \item pmf of a d.r.v. $X$ takes values from the same space $\Omega_X$ and a probability to each between 0 and 1 inclusive
    \[
    p_X:\Omega_X \to [0,1]
    \]
    \item $p_X$ is the PMF for random variable $X$, we use little $p$ not big $P$
    \item $p_X \equiv p(x)$ 
    \item $\Omega_X$ is the sample space or support of $X$
    \item $[0,1]$ is the interval of real numbers from 0 to 1 inclusive
    \item $\to$ indicates the function maps elements from the domain $\Omega_X$ to the range $[0,1]$
    \item What can we tell from this definition? 1. the pmf is non-negative and at-most 1. $0 \leq p_X \leq 1$. This \\
    \[
        p_X(k) = P(X = k)
    \]
    \item For each value $k$ in $\Omega_X$, $p_X(k)$ gives $\Pr$ that $X$ takes on $k$
\end{itemize}
\subsubsection*{Properties}
\begin{enumerate}
    \item $0 \leq p_X \leq 1$: $p_X$ is always non-negative and at most 1. This is basic intuitive understanding!
    \item $\sum p(x) = 1$ the sum of all probabilities of all outcomes = 1, that is the sum taken over all $x$ in $\Omega_X$ \\
    This formalizes the idea that at least 1 outcome must occur, and all outcomes are accounted for. Importantly, it enables the complement probability! Because we sum to 1, then the complement probability e.g. $\Pr[\text{not } A]$ = $1 - \Pr[A]$.
    \item For discrete random variables, $p(x) = 0$ for all $x$ not in the sample space $\Omega_X$ \\
    This formalizes the idea that impossible outcomes have zero probability, helps define the support/range of the distribution, allows us to extend the domain of the pmf. \\
    E.g. for a dice roll: $\Omega_X = \{1,2,3,4,5,6\}$, if $p(x) = 0$ then $p(0), p(7), p(3.5) = 0$ the probability we roll a 0 is 0.
\end{enumerate}

\begin{example}
adf
\end{example}



% Discrete Random Variable

% The capital X is commonly used to denote a random variable in probability theory. It can represent either a discrete or continuous random variable, depending on the context.
% Let me explain how a discrete random variable works:
% A discrete random variable X is a variable that can only take on a countable number of distinct values. These values are often (but not always) integers. Each possible value of X has an associated probability of occurring.
% Key characteristics of discrete random variables:

% Countable outcomes: The variable can only take on specific, separated values.
% Probability mass function (PMF): This function, often denoted as P(X = x), gives the probability that X takes on a specific value x.
% Cumulative distribution function (CDF): This function, denoted as F(x), gives the probability that X is less than or equal to x.
% Discrete jumps: The CDF of a discrete random variable has "jumps" at the possible values of X.
% Sum of probabilities: The sum of probabilities for all possible outcomes must equal 1.

% Example:
% Let's consider a simple example - rolling a fair six-sided die. Here, X could represent the outcome of the roll:

% X can take values 1, 2, 3, 4, 5, or 6
% The probability of each outcome is 1/6
% P(X = 3) = 1/6 (probability of rolling a 3)
% F(3) = P(X ≤ 3) = 1/2 (probability of rolling 3 or less)

% Expectation
