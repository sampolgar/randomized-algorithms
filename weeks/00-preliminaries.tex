\newcommand{\pmf}{p_X}

\begin{definition}
    Discrete Random Variable: 
\end{definition}

\begin{definition}
    Probability Mass Function of
    \[
    X \; \text{as} \;\;p_X:\Omega_X \to [0,1] \qquad  p_X(k) = P(X = k)
    \]
\end{definition}
\begin{itemize}
    \item pmf of a d.r.v. $X$ takes values from the same space $\Omega_X$ and a probability to each between 0 and 1 inclusive
    \[
    p_X:\Omega_X \to [0,1]
    \]
    \item $p_X$ is the PMF for random variable $X$, we use little $p$ not big $P$
    \item $p_X \equiv p(x)$ 
    \item $\Omega_X$ is the sample space or support of $X$
    \item $[0,1]$ is the interval of real numbers from 0 to 1 inclusive
    \item $\to$ indicates the function maps elements from the domain $\Omega_X$ to the range $[0,1]$
    \item What can we tell from this definition? 1. the pmf is non-negative and at-most 1. $0 \leq p_X \leq 1$. This \\
    \[
        p_X(k) = P(X = k)
    \]
    \item For each value $k$ in $\Omega_X$, $p_X(k)$ gives $\Pr$ that $X$ takes on $k$
\end{itemize}
\subsubsection*{Properties}
\begin{enumerate}
    \item $0 \leq p_X \leq 1$: $p_X$ is always non-negative and at most 1. This is basic intuitive understanding!
    \item $\sum p(x) = 1$ the sum of all probabilities of all outcomes = 1, that is the sum taken over all $x$ in $\Omega_X$ \\
    This formalizes the idea that at least 1 outcome must occur, and all outcomes are accounted for. Importantly, it enables the complement probability! Because we sum to 1, then the complement probability e.g. $\Pr[\text{not } A]$ = $1 - \Pr[A]$.
    \item For discrete random variables, $p(x) = 0$ for all $x$ not in the sample space $\Omega_X$ \\
    This formalizes the idea that impossible outcomes have zero probability, helps define the support/range of the distribution, allows us to extend the domain of the pmf. \\
    E.g. for a dice roll: $\Omega_X = \{1,2,3,4,5,6\}$, if $p(x) = 0$ then $p(0), p(7), p(3.5) = 0$ the probability we roll a 0 is 0.
\end{enumerate}

\section*{Mental Models}
\subsubsection*{Binomial Distribution}`'
Example: Flipping a fair coin 10 times and counting the number of heads. The number of heads is a binomial random variable.
Characteristics: Fixed number of trials, 2 possible outcomes, constant probability of each trial, count successes
Applications: number of defective items in a batch, number of customers making a purchase out of a total number in a store, number of opened emails in a campaign.

\subsubsection*{Poisson Distribution}
Example: Number of customers arriving at a store per hour, number of calls received by a call center per day, defects in a length of fabric
Characteristics: 

$X ~ Poi(\lambda)$
$X \in \{0, 1, \dots, \}$
$\Pr[X = k] = e^{-\lambda} \frac{\lambda^k}{k!}$

insert distribution graph here

\subsubsection*{Gaussian}

\subsubsection*{Bernoulli random variables}

\subsubsection*{Moment Generating Function}

\subsubsection*{Central Limit Theorem}