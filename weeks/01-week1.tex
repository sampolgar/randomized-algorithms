\newcommand{\Exp}{\mathbb{E}}

\begin{definition}
    Expected Value
\end{definition}
Outcomes of an experiment or random process are real numbers $a_1, \dots a_n$ with probabilities $p_1, \dots, p_n$
\[ 
    \sum_{k=1}^{n} a_kp_k = a_1p_1 + \dots + a_np_n
\]

\begin{example}
Lottery example: 500,000 people pay \$5 with winners 1 x \$1,000,000, 10 x \$1,000, 1000 x \$500, 10,000 x \$10. What's the expected value of the ticket? \\
$p_k = \frac{1}{500,000}$ for each $k = 1, \dots, 500000$ $p_k$ is the probability of each outcome occuring. $a_i$ is the net gain for a ticket $a_i$ where $a_1 = 999,995$ the net gain for winning minus the \$5 cost. 
2nd prize $a_2, \dots, a_{11} = 995$, 3rd = $a_{12}, \dots, a_{1011} = 495$ 4th = $a_{1011}, \dots, a_{1012} = 10$, 5th = $a_{1012}, \dots, a_{11011} = 10$, remainder $a_{11011}, \dots, a_{500000} = -5$
Expected value of a ticket is 
\[
    \sum_{k=1}^{500000} a_kp_k. \; Given \; p_k = \frac{1}{500000}, then \frac{1}{500000} \cdot \sum_{k=1}^{500000} a_k
\]
\[
    \frac{1}{500000}(999,995 + 10 \cdot 995 + 100 \cdot 495 + 10000 \cdot 5 + (-5) \cdot 488989) = -1.78
\]
A person who plays this lottery will on average lose 1.78 per ticket!
\end{example}


\begin{example}
    How many consecutive pairs of the same suit are expected in a deck of cards?
    \begin{enumerate}
        \item Define Indicator variables: Deck of cards = 52, 13 each suit. Let $X_i = 1$ if $i$ and $i+1$ are same suit, 0 otherwise.
        \item Calculate $P(X_i = 1)$. First $P$ = $\frac{1}{52}$, Second card matches first = $\frac{12}{51}$ for each $i$.
        \item Linearity of expectation: Total no. pairs
    \end{enumerate}

\end{example}

















Experiment
Shuffle a deck of cards, go through in order. How many times do 2 consecutive cards have the same suit?

\subsection{Linearity of expectation}
...The sum of each little thing
\[\Exp [ X + Y ] = \Exp[X] + \Exp[Y]\]
No assumption of independence or anything. Surprisingly useful. 

Prove
Expectation of selecting a card of type 1
1/13

sum of all expecations 
Xi depends on 2 cards. What's Pr

Probability of 
52 C 13
E[Xi] = Pr[Xi = 1] 


Note: difference between Expectation and Probability.
Probability = the likelihood of the event e.g. selecting 2 consecutive suit cards from a deck of 52
Expectation = the average outcome. Multiply each outcome by it's probability. 

\todo{need to do this}

\todo{discuss difference between}
Monte Carlo, Las Vegas
Running time
Output quality



Question
I have an array with n = 100
index of an even number
what is time complexity of getting even numbers
Theta n

- why isn't this constant? because you can create an algorithm that only selects 

"on expectation", the las Vegas

exepcted time 
For all expectation [Ta] = sum from infiintiy i = 1, i Pr [takes i attempts to find an even number]
\[
    \Exp[T_a] = \sum_{i=1}^{\infty} i \cdot \Pr[\text{takes $i$ attempts to find an even number}]
\]
\[
    = \sum_{i=1}^{\infty} \frac{i}{2^i} = O(1)    
\]


Why Randomization?


Faster, Simpler Algo'S
- miller rabin, it's a monte carlo algo. 
Runs in $O tilde n^2$



Algos
Quicksort
Expected running time 
Is it Las Vegas or Monte Carlo?
It's always going to return the sorted array, so it's Las Vegas.
Proof:
T(n) = Expectations[runtime on array size n]
T(n) = E[T(|A1|)] + E[T(|A2|)] + O(n)
We know |A1| + |A2| = n-1 |Why n-1? 

Expected time analysis vs worst case.
expected time analysis: for randomized algorithms
average time analysis: using input from a known probability distribution
amortized analysis: reusing algorithm on a sequence of inputs, and look at the worst-case sequence of input for the algorithm divided by the length of the sequence








\section*{Tutorial 1}
\subsubsection*{Expectation, Discrete Random Variable}
\begin{itemize}
    \item $E[X]$ = expectation of random variable $X$
    \item $X$ is a discrete random variable having probability mass function $p(x)$, then $E[X]$ is defined by 
    \[
        E[X] = \sum_{x:p(x)>0}^{} xp(x)
    \]
    \item Probability Mass Function (PMF) $p(x)$ gives $\Pr$ that a discrete random variable $X$ is equal to some value $x$
    \item PMF: $p(x) \geqq 0$ for all $x$, $\sum_{}^{}p(x) = 1$ the sum of probabilities over all possible values
\end{itemize}
\subsubsection*{Expectation, Continuous Random Variable}
\begin{itemize}
    \item $X$ is a continuous random variable having probability mass function $f(x)$, then $E[X]$ is defined by 
    \[
        E[X] = \int_{x:p(x)>0}^{} xf(x)d(x)
    \]
    \item PDF = probability density function
\end{itemize}

\subsubsection*{Variance}
\begin{itemize}
    \item Def: average of the squared differences from the mean
           \[
                X: Var(X) = \Exp[(X-\Exp[X]^2)] \; \text{By Linearity =} \Exp[X^2] - (E[X])^2
            \]      
    \item $Var(aX + b) = a^2Var(X)$ for constants $a$ and $b$
    \item For independent random variables: $Var(X + Y) = Var(X) + Var(Y)$
    \item Indicates how much a data point deviates from the mean, low variance = close to mean, high variance = far from mean, wider range
    \item Example: population $\sigma^2 = \sum(X - \mu)^2 / N.$ Sample: $s^2 = \sum  \tfrac{(X - X)^2}{n-1}$.
    \item $\sigma^2 / s^2$ is the variance, $X$ is each value in a data set, $\mu or X$ is the mean, $N or n$ is number of data points 
    \item Expressed as squared units of the original data, sometimes confusing
    \item Standard deviation is the square root of variance, in the same units of the original data
    \item 
\end{itemize}

Problem 1
Consider a deck of 4$n$ cards with 'S', 'H', 'D', 'C', after shuffled randomly, what's the expected number of consecutive pairs of the same suit.
\begin{enumerate}
    \item Define Indicator Variable $X_i$ for each iteration required. We have $4n-1$ because $4n$ cards, minus 1 match because the last card can't be matched with the null pointer next door. Let $S$ = shuffled card
    \[
    X_i =
    \begin{cases} 
    1 & \text{if $S_i = S_{i + 1}$ } \\
    0 & \text{else}
    \end{cases}
    \]
    \item Sum the number of consecutive pairs found, i.e., sum all 1 cases
    \[
        X = \sum_{i=1}^{4n-1} X_i \; \text{we need to find }E[X] = E[\sum_{i=1}^{4n-1} X_i] \;
    \]
    By linearity
    \[
         = \;  \sum_{i=1}^{4n-1} E[X_i] \; = \;  \sum_{i=1}^{4n-1} \frac{\text{no. cards in a suit}}{\text{no. cards in a deck}}
    \]
    We need to find $E[X_i]$ for each $i$ = $\Pr$ that $S_i = S_{i + 1}$:
    \[
        \Pr(X_i = 1) = \frac{n-1}{4n-1} \; = E[X_i] = \frac{n-1}{4n-1}\; 
    \]
\end{enumerate}

Problem 2
Similar

Problem 3
Similar
Variance part explained here https://claude.ai/chat/9210ca3a-e032-4137-80b1-455acfe9835a

Problem 4
explained here https://claude.ai/chat/fc539a53-9b45-4e59-9638-fcd3e8532612 

Problem 5




\section*{Quiz 0}
Q1: Handshaking Lemma states if $G = (V,E)$ is an undirected graph, $\sum_{v \in V}^{} deg v$ is equal to: $2|E|$. \\
Recall E0--v--0E 

Q2: Linearity of expectation means if $X, Y$ are 2 arbitrary random variables and $a, b$ are 2 arbitrary random numbers then $\Exp [aX + bY] = \Exp[aX] + \Exp[bY] = a\Exp[X] + b\Exp[Y]$ if:
\begin{enumerate}
    \item as long as both expectations are defined
    \item only if X, Y are independent
    \item only if a, b are positive
    \item only if X, Y are uncorrelated
\end{enumerate}